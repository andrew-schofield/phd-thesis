\svnid{$Id$}
\chapter{Discussion and Completed Model}
\label{chap:completedmodel}
%Therefore I have integrated microbiological techniques with mathematical modelling principles to allow us to understand the system in a way previously not possible. I have constructed a mathematical model which describes \Nsm{} respiration \textit{in silico} and can produce predictions for the system \textit{in vivo}. To parametrise this model I used a novel integrative scheme where a standard Bayesian fitting methodology is interleaved with iterative experimental data collection on progressively more complex respiratory scenarios.

\begin{table}[tbp]
\begin{center}
\begin{tabular}{>{\centering}m{1.4cm}>{\centering}m{6.1cm}>{\centering}m{2.7cm}>{\centering}m{2.5cm}}
\toprule
\textbf{Symbol} & \textbf{Description} & \textbf{Mean Value} & \textbf{$\sigma$}
\tabularnewline
\midrule
$k_1$ & Rate constant for O$_{\textrm{2}}$ reduction by reduced \cbbthree{} & $403.51~\mu M^{-1} s^{-1}$ & $27.59$
\tabularnewline\noalign{\smallskip}\hline\noalign{\smallskip}

$k_3$ & Rate constant for \cbbthree{} reduction by cytochrome pool & $4.58~\mu M^{-1} s^{-1}$ & $0.436$
\tabularnewline\noalign{\smallskip}\hline\noalign{\smallskip}

$l_1$ & Rate constant for NO reduction by reduced NorB & $6.42~\mu M^{-1} s^{-1}$ & 2.33
\tabularnewline\noalign{\smallskip}\hline\noalign{\smallskip}

$l_3$ & Rate constant for NorB reduction by quinone pool & $0.096~\mu M^{-1} s^{-1}$ & $0.025$
\tabularnewline\noalign{\smallskip}\hline\noalign{\smallskip}

$m_1$ & Rate constant for NO$_{\textrm{2}}^{\textrm{-}}$ reduction by reduced AniA & $0.175~\mu M^{-1} s^{-1}$ & $0.087$
\tabularnewline\noalign{\smallskip}\hline\noalign{\smallskip}

$m_3$ & Rate constant for AniA reduction by cytochrome pool & $4.79~\mu M^{-1}s^{-1}$ & $0.042$
\tabularnewline\noalign{\smallskip}\hline\noalign{\smallskip}

$k_5$ & Rate constant for \cbbthree{} inhibition by NO & $\approx66000~\mu M ^{-1} s ^{-1}$ & Indeterminate
\tabularnewline\noalign{\smallskip}\hline\noalign{\smallskip}

$k_6$ & Rate constant for recovery of NO inhibited \cbbthree{} & $1.59~s^{-1}$ & $0.527$
\tabularnewline\noalign{\smallskip}\hline\noalign{\smallskip}

$\beta$ & Rate constant for passive diffusion in of O$_{\textrm{2}}$ & $0.00014~\mu M^{-1} s^{-1}$ & $4.7\times10^6$
\tabularnewline\noalign{\smallskip}\hline\noalign{\smallskip}

$K_O$ & Saturation O$_{\textrm{2}}$ level & $48~\mu M$ & $0$
\tabularnewline\noalign{\smallskip}\hline\noalign{\smallskip}

$g$ & Rate of electrons in from NADH & $0.085~s^{-1}$ & $0.0078$
\tabularnewline\noalign{\smallskip}\hline\noalign{\smallskip}

$f$ & Rate constant for reduction of cytochromes by quinones & $0.771~\mu M^{-1}s^{-1}$ & $0.096$
\tabularnewline\noalign{\smallskip}\hline\noalign{\smallskip}

$\gamma$ & Spontaneous loss of NO & $0.0024~\mu Ms^{-1}$ & $8.8\times10^5$
\tabularnewline\noalign{\smallskip}\hline\noalign{\smallskip}

$Q$ & Concentration of quinones & $34.31~\mu M$ & $5.49$
\tabularnewline\noalign{\smallskip}\hline\noalign{\smallskip}

$X$ & Concentration of cytochromes & $2.81~\mu M$ & $0.595$
\tabularnewline\noalign{\smallskip}\hline\noalign{\smallskip}

$A$ & Concentration of AniA & $0.704~\mu M$ & $0.3$
\tabularnewline\noalign{\smallskip}\hline\noalign{\smallskip}

$B$ & Concentration of NorB & $7.8~\mu M$ & $3.63$
\tabularnewline\noalign{\smallskip}\hline\noalign{\smallskip}

$C$ & Concentration of \cbbthree{} & $1.3~\mu M$ & $0.259$
\tabularnewline
\bottomrule
\end{tabular}
\caption[Model parameters]{{\bf Model parameters.} This table shows all the parameter values that have been obtained from the extant literature, or interpolated from preliminary experiments done during the course of this work. These values represent the initial data that is used to populate the model, from which all subsequent parameter sets are generated. For values that show concentrations of components, they represent the value for a culture with $OD_{600}=1.00$.
\label{tab:final_parameters}}
\end{center}
\end{table}

\section{Amalgamation of cytochromes}