\chapter{Model - Construction and Parameters}
\section{Construction}

The model was constructed based on existing knowledge of the respiratory chain in \Nsm{} from the electron transport chain shown in \ref{fig:etc}. I made no \textit{a priori} assumptions about separation of time-scales that would permit the use of Michaelis-Menton kinetics, as the rates of intermediate reaction steps are not known.
The model was generated as a set of ordinary differential equations which describe the bulk-average concentration of substrates, products, enzymes and their activity. I have made no assumptions about the bacterial population structure and as such any stochastic effects are ignored.

\subsection{Converting Biological Reactions into Differential Equations}
Where the reaction is describing a chemical process, the rate constant is given above the arrow, and the relevant enzyme shown in parentheses. Where the reaction is showing the addition of electrons (reduction), this is denoted by $e^-$ below the arrow, the rate constant above, and the source of electrons in parentheses.

The equation that gives the change in oxygen concentration is

\begin{eqnarray}
&\frac{d[O_2]}{dt} = \beta(1-[O_2]/K_O) - k_{1}[C_a][O_2] \nonumber \\\nonumber \\
&\xrightarrow{\beta} \mathbf{O_2} \xrightarrow{k_1~(C_a)} \textnormal{H$_\textnormal{2}$O}
\label{eq:oxygen}
\end{eqnarray}

where $\beta$ is the rate of passive diffusion of \cOxygen{} into the electrode chamber. This is inversely proportional to oxygen concentration in the chamber, and limited to the oxygen saturation concentration, $K_O$. This component of the equation is required to account for a peculiarity of the experimental set-up, whereby the rate of diffusion of oxygen into the system depends on the density of the bacterial culture, and is not insignificant. $k_{1}$ is the rate of reduction of oxygen by the oxygen reductase \cbbthree{}. This rate depends on the concentration of reduced (i.e. active) \cbbthree{}, $C_a$ and the concentration of \cOxygen{}.

The equation for describing \cNO{} concentration changes is more complex as \cNO{} has a number of additional interactions in comparison to \cOxygen{}. \cNO{} also interacts with \cbbthree{}, in addition to being reduced from \cNitrite{}, reduced to \cNtwoO{} and spontaneously lost from the electrode chamber. Currently this is the equation being used to model \cNO{} concentration.

\begin{eqnarray}
&\frac{d[NO]}{dt} = m_{1}[NO_2^-][A_a] - l_1[NO][B_a] - k_5[C_a][NO] + k_6 [C_X] - \gamma[NO] \nonumber \\\nonumber \\
&\textnormal{NO$_\textnormal{2}^\textnormal{-}$} \xrightarrow{m_1~(A_a)} \mathbf{NO} \xrightarrow{l_1~(B_a)} \textnormal{N$_\textnormal{2}$O} \nonumber \\
&\mathbf{NO} + \textnormal{C$_\textnormal{a}$} \xrightarrow{k_5} \textnormal{NO}\mhyphen \textnormal{C$_\textnormal{X}$} \xrightarrow{k_6} \mathbf{NO} + \textnormal{C$_\textnormal{a}$} \nonumber \\
&\xrightarrow{\gamma} \mathbf{NO}
\label{eq:no}
\end{eqnarray}

The synthesis of \cNO{} is modelled by $m_{1}$ which is the rate of \cNitrite{} reduction by reduced (active) AniA. This also depends on the concentration of \cNitrite{} and reduced AniA ($A_a$). The reduction of \cNO{} requires $l_1$ which is the rate of reduction of \cNO{} by reduced (active) NorB. This depends on the concentration of \cNO{} and reduced NorB ($B_a$). Inhibition of \cbbthree{} by \cNO{} is modelled by the \engordnumber{3} component of the equation. $k_5$ is the rate of inhibition of \cbbthree{} by \cNO{}. $k_6$ is the rate of recovery of inhibited \cbbthree{}. $\gamma$ is the rate of spontaneous loss of \cNO{} from the electrode chamber.

The reduction of nitrite is modelled by this equation

\begin{eqnarray}
&\frac{d[NO_2^-]}{dt} = - m_{1}[NO_2^-][A_a] \nonumber \\\nonumber \\
&\mathbf{NO_2^-} \xrightarrow{m_1~(A_a)} \textnormal{NO}
\label{eq:nitrite}
\end{eqnarray}

where $m_{1}$ is the rate of reduction of \cNitrite{} by reduced (active) AniA ($A_a$).

In addition to the rate of change of concentration of the respiratory substrates, the model also contains information about the state of the quinone pool, which is the upstream source of electrons into the respiratory chain. This is important because this affects the rate of reduction of the various enzymes which perform the substrate reductions. The equation for modelling the change in reduction state (activity) of the quinone pool is

\begin{eqnarray}
&\frac{d[Q_a]}{dt} = g([Q] - [Q_a]) - l_3[Q_a]([B] - [B_a]) - f[Q_a]([X]-[E]) \nonumber \\\nonumber \\
&\xrightarrow[e^-]{g} \mathbf{Q_a} \nonumber \\
&\textnormal{B$_\textnormal{i}$} \xrightarrow[e^-]{l_3~(\mathbf{Q_a})} \textnormal{B$_\textnormal{a}$} \nonumber \\
&\textnormal{X-E} \xrightarrow[e^-]{f~(\mathbf{Q_a})} \textnormal{E}
\label{eq:quinones}
\end{eqnarray}

$Q_a$ is the reduced quinone, and $Q$ the total concentration of quinones in the system. $g$ represents the rate of flow of electrons into the quinone pool from NADH. The rate of reduction of NorB by active quinones is given by $l_3$. NorB and reduced NorB are given by $B$ and $B_a$ respectively. As the quinones also reduce the cytochromes, this also needs to be modelled. $f$ denotes the rate of reduction of cytochromes by the active quinones. Cytochromes and reduced cytochromes are given by $X$ and $E$ respectively.

Given that the concentration of active cytochromes changes, due to reduction by the quinone pool and oxidation by the downstream enzymes, and this concentration is a parameter in (\ref{eq:quinones}), it also needs to be included in the model, and this is given by the following equation

\begin{eqnarray}
&\frac{d[E]}{dt} = -k_3([C] - [C_a] - [C_X])[E]  - m_3([A] - [A_a])[E] + f[Q_a]([X]-[E]) \nonumber \\\nonumber \\
&\textnormal{C$_\textnormal{i}$} \xrightarrow[e^-]{k_3~(\mathbf{E})} \textnormal{C$_\textnormal{a}$} \nonumber \\
&\textnormal{A$_\textnormal{i}$} \xrightarrow[e^-]{m_3~(\mathbf{E})} \textnormal{A$_\textnormal{a}$} \nonumber \\
&\textnormal{X-E} \xrightarrow[e^-]{f~(Q_a)} \mathbf{E}
\label{eq:cytochromes}
\end{eqnarray}

where $k_3$ is the rate of reduction of the cytochrome c oxygen reductase (\cbbthree{}) by the quinone pool (via \textit{c$_{\textrm{x}}$} \& \textit{c$_{\textrm{4}}$}). $C$, $C_a$ and $C_X$ represent the overall concentration of \cbbthree{}, reduced (active) \cbbthree{} and NO inhibited \cbbthree{} respectively. $m_3$ is the rate of reduction of AniA by the cytochrome pool (via \textit{c$_{\textrm{5}}$}). The concentration of active cytochromes increases by their reduction by the quinone pool.

To model the changes in concentration of the individual enzymes, \cbbthree{}, AniA and NorB, the following equations are used:

\begin{eqnarray}
&\frac{d[C_a]}{dt} = k_3([C] - [C_a] - [C_X])[E] - k_{1}[C_a][O_2] - k_5[C_a][NO] \nonumber \\\nonumber \\
&\textnormal{C$_\textnormal{i}$} \xrightarrow[e^-]{k_3~(E)} \mathbf{C_a} \nonumber \\
&\textnormal{O$_\textnormal{2}$} \xrightarrow{k_1~(\mathbf{C_a})} \textnormal{H$_\textnormal{2}$O} \nonumber \\
&\textnormal{NO} + \mathbf{C_a} \xrightarrow{k_5} \textnormal{NO}\mhyphen \textnormal{C$_\textnormal{X}$}
\label{eq:active_cbb3}
\end{eqnarray}

This equation models the concentration of reduced (active) \cbbthree{}, and the following equation models the concentration of \cbbthree{} that has been inhibited by \cNO{}.

\begin{eqnarray}
&\frac{d[C_X]}{dt} = k_5[C_a][NO] - k_6 [C_X] \nonumber \\\nonumber \\
&\textnormal{NO} + \textnormal{C$_\textnormal{a}$} \xrightarrow{k_5} \textnormal{NO}\mhyphen \mathbf{C_X} \xrightarrow{k_6} \textnormal{NO} + \textnormal{C$_\textnormal{a}$}
\label{eq:NO inhibited_cbb3}
\end{eqnarray}

Reduced (active) AniA concentrations are modelled by this equation

\begin{eqnarray}
&\frac{d[A_a]}{dt} = m_3([A] - [A_a])[E]- m_{1}[NO_2^-][A_a] \nonumber \\\nonumber \\
&\textnormal{A$_\textnormal{i}$} \xrightarrow[e^-]{m_3~(E)} \mathbf{A_a} \nonumber \\
&\textnormal{NO$_\textnormal{2}^\textnormal{-}$} \xrightarrow{m_1~(\mathbf{A_a})} \textnormal{NO}
\label{eq:active_ania}
\end{eqnarray}

and reduced (active) NorB concentrations are modelled by this equation

\begin{eqnarray}
&\frac{d[B_a]}{dt} = l_3[Q_a]([B] - [B_a]) - l_1[NO][B_a] \nonumber \\\nonumber \\
&\textnormal{B$_\textnormal{i}$} \xrightarrow[e^-]{l_3~(Q_a)} \mathbf{B_a} \nonumber \\
&\textnormal{NO} \xrightarrow{l_1~(\mathbf{B_a})} \textnormal{N$_\textnormal{2}$O}
\label{eq:active_norb}
\end{eqnarray}

By keeping the quantities involved in their original state and not making any assumption about time-scale separation I am able to make predictions regarding the transient oxidation states of the various components. These are potentially experimentally accessible and appear to be crucial for the dynamic response of the chain in different environments.

The model contains no implied information about cell density. This means the values for various component concentrations will differ between experiments. Initially the optical density of cultures was used to determine the cell density however experiments proved that this was not a completely reliable proxy for cell density as this also includes dead cells. Using optical density as a cell density proxy should haven given linear relations between cell densities and reaction rates, however this proved not to be the case, with rates of oxygen reduction different between cultures with the same optical density. Therefore where possible, any normalisation that was carried out used the initial oxygen reduction rate as a relative indicator.

\subsection{Assumptions and their Justifications}
I have made a number of assumptions regarding the kinetics and reactions taking place in the model.
\begin{enumerate}
 \item {\bf I have assumed that NO inhibits the reduced \cbbthree{} and not the oxidised form, since I wouldn't expect Nitric oxide to bind to an inactive enzyme.} This is corroborated by \citet{Giuffre2000}, who show significant levels of inhibition of reduced cytochrome. They do also however observe low levels of inhibition of the oxidised enzyme also. Their experiments used cytochrome c oxidase (aa3) rather than \cbbthree{}, but I believe this assumption still stands as the enzymes are of the same family.
 \item {\bf No backwards reactions.}
 \item {\bf No Michaelis-Menton kinetics.}
 \item {\bf All cytochromes can be modelled as one.}
 \item {\bf Laz and $c_5$ effects on AniA and \cbbthree{} respectively can be ignored.} They are not the prime electron donors to their terminal reductases and contribute very little overall to the reduction\cite{Deeudom2007}.
\end{enumerate}

%Also, values are in the right ball park even though they are using aa3 rather than \cbbthree. $10^8 M ^{-1} s ^{-1}$ against around $50 \mu M ^{-1} s ^{-1}$.

\section{Parameters and their Prior Distributions}

None of the rate constants or concentrations which were required for this model have previously been determined for \Nsm, so values from other similar organisms had to be used instead. In some cases there appears to be no data in the literature regarding values of particular components. Table \ref{tab:ps} lists the values that have been obtained from the literature.

\begin{table}[ht!]
\begin{center}
\begin{tabular}{ccc}
\toprule
\textbf{Symbol} & \textbf{Description} & \textbf{Value}\\
\midrule
$k_1$ & Rate of O$_{\textrm{2}}$ reduction by reduced \cbbthree{} & $415\mu M^{-1} s^{-1}$ \\
$k_3$ & Rate of \cbbthree{} reduction by cytochrome pool & $3\mu M^{-1} s^{-1}$\\ 
$l_1$ & Rate of NO reduction by reduced NorB\\
$l_3$ & Rate of NorB reduction by quinone pool\\
$m_1$ & Rate of NO$_{\textrm{2}}^{\textrm{-}}$ reduction by reduced AniA\\
$m_3$ & Rate of AniA reduction by cytochrome pool & $4.8\pm0.2 \mu M^{-1}s^{-1}$\\
$k_5$ & Rate of \cbbthree{} inhibition by NO & $10^8 M ^{-1} s ^{-1}$\\
$k_6$ & Rate of recovery of NO inhibited \cbbthree{}\\
$\beta$ & Rate of passive diffusion in of O$_{\textrm{2}}$\\
$K_O$ & Saturation O$_{\textrm{2}}$ level & $126\mu M$\\
$g$ & Rate of electrons in from NADH\\
$f$ & Rate of reduction of cytochromes by quinones\\
$\gamma$ & Spontaneous loss of NO\\
$Q$ & Concentration of quinones & $0.3\mu M$\\
$X$ & Concentration of cytochromes & 4000nM - \cbbthree{}\\   
$A$ & Concentration of AniA\\
$B$ & Concentration of NorB\\
$C$ & Concentration of \cbbthree{} & 30nM \\
\bottomrule
\end{tabular}
\caption{Model parameters
\label{tab:ps}}
\end{center}
\end{table}

\subsection*{Variables}
\subsubsection*{$\mathbf{O_2}$ {\bf- Oxygen concentration}}
This variable is always obtained directly from the experimental dataset as it indicates the starting point for oxygen in the model. It is always set to the first oxygen data point in the dataset and has no prior distribution. It is usually a fixed value, except in cases where the dataset indicates measurement artefacts.

\subsubsection*{$\mathbf{NO}$ {\bf- Nitric oxide concentration}}
As for Oxygen concentration, this variable is simply obtained from the dataset and the same conditions apply.

\subsubsection*{$\mathbf{NO_2^-}$ {\bf- Nitrite concentration}}
Nitrite concentration is also handled in the same way as the oxygen and nitric oxide concentrations.

\subsubsection*{$\mathbf{E}$ {\bf- Reduced cytochrome concentration}}
Unknown at start of simulation. Assume close to zero, and certainly less than total. The best prior to use would be the value after a couple of simulation seconds, however this would no longer be a prior.

\subsubsection*{$\mathbf{A_a}$ {\bf- Reduced AniA}}
Unknown at start of simulation. Assume close to zero, and certainly less than total. The best prior to use would be the value after a couple of simulation seconds, however this would no longer be a prior.

\subsubsection*{$\mathbf{B_a}$ {\bf- Reduced NorB}}
Unknown at start of simulation. Assume close to zero, and certainly less than total. The best prior to use would be the value after a couple of simulation seconds, however this would no longer be a prior.

\subsubsection*{$\mathbf{C_a}$ {\bf- Reduced \cbbthree{}}}
Unknown at start of simulation. Assume close to zero, and certainly less than total. The best prior to use would be the value after a couple of simulation seconds, however this would no longer be a prior.

\subsubsection*{$\mathbf{C_X}$ {\bf- Reversibly NO inhibited \cbbthree{}}}
Unknown at start of simulation. Assume close to zero, and certainly less than total. The best prior to use would be the value after a couple of simulation seconds, however this would no longer be a prior.

\subsubsection*{$\mathbf{Q_a}$ {\bf- Reduced Quinones}}
Unknown at start of simulation. Assume close to zero, and certainly less than total. The best prior to use would be the value after a couple of simulation seconds, however this would no longer be a prior.

\subsection*{Parameters}
\subsubsection*{$\mathbf{k_1}$ {\bf- Rate of O$_{\textrm{2}}$ reduction by reduced \cbbthree{}}}
\citet{Preisig1996} show that \textit{B. japonicum} \cbbthree{} (\textit{fixNOQP}) has a $K_m$ of $55.7 \pm 24.2$ nM $\mathrm{O}_2$, and $V_{max}$ $37.4 \pm 9.2$ $\mathrm{nmol O}_2 \mathrm{min}^{-1} \mathrm{mg}^{-1}$.\\
Given $v = \frac{V_{max}[S]}{K_m+[S]}$
Then at high $O_2$ rate is:
$v = \frac{37.4\times 100,000}{55.7+100,000} = 37.4nmol^{-1}min{-1}mg{-1} = 0.000622986\mu mol^{-1}s{-1}mg{-1}$\\

A value for $k_1$, was calculated by using the $\mathrm{K}_{cat}$ value from \textit{Pseudomonas stutzeri}\cite{Forte2001}, and the $\textrm{k}_m$ value from \textit{Neisseria lactamica}\cite{Hunter2007}, which are $166s^{-1}$ and $0.4\mu M$ respectively. $k_1$ can be calculated as $\frac{166s^{-1}}{0.4\mu M} = 415\mu M^{-1} s^{-1}$.

\subsubsection*{$\mathbf{k_3}$ {\bf- Rate of \cbbthree{} reduction by cytochrome pool}}
This was calculated from values obtained from the maximum reduction rate of \cbbthree{} by cytochrome $c_4$ in \textit{Vibrio cholerae}\cite{Chang2010}. A rate of 300 electrons transported per second was observed with a cytochrome $c_4$ concentration of $100\mu M$. This concentration was not saturating, but there appears to be a linear relationship between rate and concentration. I assume that 1 electron equals 1 reduction of \cbbthree{}, thus the rate of reduction of \cbbthree{} by cytochromes is $\frac{300s^{-1}}{100\mu M} = 3\mu M^{-1} s^{-1}$.

\subsubsection*{$\mathbf{l_1}$ {\bf- Rate of NO reduction by reduced NorB}}
240 and 256 nanomoles NO reduced per minute per OD600 unit \citet{Barth2009}.

This needs fixing. 50\% dry weight, rather than 15\% wet weight.\\
Observed rates of NO reduction by \citet{Rock2007} give $54 \pm 6 \mathrm{nmol min}^{-1} \mathrm{mg}^{-1}$. This is in whole cells however.
$\approx 10\mathrm{nmol s}^{-1}$ in an $OD_{600} = 1$ culture. Converting to molar gives $2\mu\mathrm{M s}^{-1}$. 

\subsubsection*{$\mathbf{l_3}$ {\bf- Rate of NorB reduction by quinone pool}}
Benchmark estimate is somewhere around about $1\mu M^{-1}s^{-1}$.

\subsubsection*{$\mathbf{m_1}$ {\bf- Rate of NO$_{\textrm{2}}^{\textrm{-}}$ reduction by reduced AniA}}


\subsubsection*{$\mathbf{m_3}$ {\bf- Rate of AniA reduction by cytochrome pool}}
This value is the observed electron transfer rate between the equivalent cytochrome and nitrite reductase from \textit{Achromobacter xylosoxidans}. A value of $4.8\pm0.2 \mu M^{-1}s^{-1}$ was observed during stopped-flow experiments\cite{Nojiri2009}.

\subsubsection*{$\mathbf{k_5}$ {\bf- Rate of \cbbthree{} inhibition by NO}}
\citet{Giuffre2000} and \citet{Blackmore1991} showed with cytochrome \textit{c} oxidase that NO could bind reversibly and inhibit the activity of the enzyme. The rate they calculated was $10^8 M ^{-1} s ^{-1}$. I assume that even though the enzyme is different, its NO binding characteristics would be similar to that of \cbbthree{} as it is of the same family.

\subsubsection*{$\mathbf{k_6}$ {\bf- Rate of recovery of NO inhibited \cbbthree{}}}
\citet{Giuffre2000} calculated a half-life of t\textonehalf $\approx 80 \mathrm{min}$.\\
$K_d$ from \citet{Rock2007} was calculated to be about 500nM, which tallies with values from $k_5$ and $k_6$.

\subsubsection*{$\beta$ {\bf- Rate of passive diffusion in of O$_{\textrm{2}}$}}
This value is highly dependent on the culture, and is in some way tied to the density of the culture, however the relationship is not known. During early experiments I noticed that oxygen diffusion was slower in high density cultures compared to those of low density, experiments to examine the relationship proved fruitless in determining any relationship. In addition this parameter is a product of the experimental set-up rather than the model itself.

\subsubsection*{$\mathbf{K_O}$ {\bf- Saturation O$_{\textrm{2}}$ level}}
This value is dependent on the particular culture being modelled, however it's prior value is usually set to $126\mu M$ as this figure was observed during experiments to determine oxygen diffusion rates into the culture.

\subsubsection*{$\mathbf{g}$ {\bf- Rate of electrons in from NADH (or rate of reduction of quinones)}}


\subsubsection*{$\mathbf{f}$ {\bf- Rate of reduction of cytochromes by quinones}}
\citet{Snyder2000} showed by reducing yeast cytochrome $\mathrm{bc}_1$ by using $25\mu\mathrm{M}$ menaquinol the rate constants were $7.9\mathrm{s}^{-1}$ for cytochrome b, and $1.55-6.9\times10^5\mathrm{M}^{-1}\mathrm{s}^{-1}$ for cytochrome $\mathrm{c}_1$ (second order). $0.155 - 0.69\mu\mathrm{M}^{-1}\mathrm{s}^{-1}$.

\subsubsection*{$\gamma$ {\bf- Spontaneous loss of NO}}


\subsubsection*{$\mathbf{Q}$ {\bf- Concentration of quinones}}
This value was calculated based on data from Hedricks et al\cite{Hedrick1986}. The protein content of the cells was assumed to be similar to that of \textit{E. coli} at 15\% of wet weight, where each cell weighed 2pg, and that there were $1\mu \textrm{mol}$ of respiratory quinones per g of bacterial protein. A culture of \textit{Neisseria meningitidis} with $OD_{600} = 1$ has $1 \times 10^9 \textrm{cells/ml}$, therefore there are 1.5nmol of quinones in 5ml culture ($5\times 10^9 \textrm{cells} \times 2\times 10^{-12} \textrm{g} \times 15\% \times 1\mu\textrm{mol/g}$), converted to molarity is $0.3\mu M$.

\subsubsection*{$\mathbf{X}$ {\bf- Concentration of cytochromes}}
Manu's thesis\cite{Deeudom2007} suggests total cytochrome concentration (inc. \cbbthree{}) to be about 4000nM.

\subsubsection*{$\mathbf{A}$ {\bf- Concentration of AniA}}
No idea, probably need to guess based on cell volume (0.6-1.0um diameter, no useful ref), 10\% of cell volume being membrane, and number of proteins in membrane.

\subsubsection*{$\mathbf{B}$ {\bf- Concentration of NorB}}
No idea, probably need to guess based on cell volume (0.6-1.0um diameter, no useful ref), 10\% of cell volume being membrane, and number of proteins in membrane.

\subsubsection*{$\mathbf{C}$ {\bf- Concentration of \cbbthree{}}}
No idea, probably need to guess based on cell volume (0.6-1.0um diameter, no useful ref), 10\% of cell volume being membrane, and number of proteins in membrane.\\
\cbbthree{} is probably 0.1-1\% of cell protein. 10\% of cell is membrane.
$15\mu g$ in 5ml based on numbers from Q above. \cbbthree{} is approximately 100KDa in molecular weight. Converting to molarity gives a concentration of approximately 30nM.