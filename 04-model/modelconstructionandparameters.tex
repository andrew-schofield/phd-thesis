\chapter{Model - Construction and Parameters}
\label{chap:model}
\section{Construction}

The model was constructed based on existing knowledge of the respiratory chain in \Nsm{} from the electron transport chain shown in Figure \ref{fig:etc} (Chapter \ref{chap:intro}). I made no \textit{a priori} assumptions about separation of time-scales that would permit the use of Michaelis-Menton kinetics, as the rates of intermediate reaction steps are not known. This approach also permits tracking of the oxidation state of all the intermediates which allows understanding and offers the potential for predictions that may be explored in future \textit{in vivo} studies.

The model was generated as a set of ordinary differential equations which describe the bulk-average concentration of substrates, products, enzymes and their activity within a well-mixed vessel. I have made no assumptions about the bacterial population structure or the variations in concentrations of substrates between the bulk media and within the bacterial cells. Stochastic effects are ignored, but they are unlikely to be of importance. Additionally protein production is largely ignored as the switching mechanism happens on a time-scale that is much shorter than the transcription and translation of new proteins, they are therefore assumed to be expressed constitutively except where stated otherwise.

\subsection{Converting Biological Reactions into Differential Equations}
The rationale for obtaining the form for each of the 9 component equations is described below. Throughout the model the reduced components, i.e. those with available electrons, are denoted as active and have a subscript $a$. Components lacking a subscript denote the total, constant, amount of a component.

\subsubsection{Respiratory Substrates}
The equation that gives the change in oxygen concentration is

\begin{equation}
\frac{d[O_2]}{dt} = \beta(1-[O_2]/K_O) - k_{1}[C_a][O_2]
\label{eq:oxygen}
\end{equation}
\begin{equation*}
\boxed{
\begin{gathered}
\emptyset\xrightarrow{\beta} O_2\\
O_2 + C_a\xrightarrow{k_1} H_{2}O + C_i
\end{gathered}
}
\end{equation*}
where $\beta$ is the rate of passive diffusion of \cOxygen{} into the electrode chamber. In isolation this term gives rise to a simple exponential input of oxygen until the saturation level is reached. One small complication encountered was that $\beta$ is itself affected by the amount of bacteria in the vessel which has implications for fitting. The rate constant $k_1$ describes the reduction of oxygen and the oxygen reductase \cbbthree{}; this rate depends on the concentration of reduced (i.e. active) \cbbthree{}, $C_a$ and the concentration of $O_2$.
%This is inversely proportional to oxygen concentration in the chamber, and limited to the oxygen saturation concentration, $K_O$. This component of the equation is required to account for a peculiarity of the experimental set-up, whereby the rate of diffusion of oxygen into the system depends on the density of the bacterial culture, and is not insignificant. $k_{1}$ is the rate of reduction of oxygen by the oxygen reductase \cbbthree{}. This rate depends on the concentration of reduced (i.e. active) \cbbthree{}, $C_a$ and the concentration of \cOxygen{}.%

The equation for describing NO concentration changes has a number of additional interactions in comparison to $\mathrm{O}_2$. NO is created by the reduction of $\mathrm{NO}_\mathrm{2}^\mathrm{-}$ by AniA, is reduced by its dedicated reductase, NorB, and converted to $\mathrm{N}_2\mathrm{O}$ which is lost from the cell, interacts with \textit{cbb$_{\textrm{3}}$}, and is also spontaneously lost from the electrode chamber. I make the assumption that the interaction with \textit{cbb$_{\textrm{3}}$} occurs only in a reversible manner, leading to an NO bound and temporarily inactive form $C_X$. There is evidence that this interaction can also lead to permanent degradation of \textit{cbb$_{\textrm{3}}$} via the formation of peroxynitrite at the terminal oxidase. This is not currently considered in this version of the model. These effects are described mathematically in the equation below in the order presented.
%The equation for describing \cNO{} concentration changes is more complex as \cNO{} has a number of additional interactions in comparison to \cOxygen{}. \cNO{} also interacts with \cbbthree{}, in addition to being reduced from \cNitrite{}, reduced to \cNtwoO{} and spontaneously lost from the electrode chamber. Currently this is the equation being used to model \cNO{} concentration.

\begin{equation}
\frac{d[NO]}{dt} = m_{1}[NO_2^-][A_a] - l_1[NO][B_a] - k_5[C_a][NO] + k_6 [C_X] - \gamma[NO]
\label{eq:no}
\end{equation}
\begin{equation*}
\boxed{ 
\begin{gathered}
NO_2^- + A_a \xrightarrow{m_1} NO + A_i\\
NO + B_a \xrightarrow{l_1} N_2O + B_i\\
NO + C_a \xrightarrow{k_5} NO\mhyphen C_X\\
NO\mhyphen C_X \xrightarrow{k_6} NO + C_a\\
\emptyset\xrightarrow{\gamma} NO
\end{gathered}
}
\end{equation*}

The rate of synthesis of NO is captured by the first term, with rate constant $m_{1}$ and depends on the both the concentration of $\mathrm{NO}_\mathrm{2}^\mathrm{-}$ and reduced AniA ($A_a$). The reduction of NO is described by the next term with the rate constant $l_1$ and also depends on the concentration of NO and reduced NorB ($B_a$). Inhibition of \textit{cbb$_{\textrm{3}}$} by NO is modelled by the $\mathrm{3}^\mathrm{rd}$ component of the equation. $k_5$ is the rate constant describing the reversible binding of NO to \textit{cbb$_{\textrm{3}}$} to form the inactive form of \textit{cbb$_{\textrm{3}}$}, $C_X$. $k_6$ is the rate of recovery of this inhibited \textit{cbb$_{\textrm{3}}$}. $\gamma$ is the spontaneous rate of loss of NO from the electrode chamber.
%The synthesis of \cNO{} is modelled by $m_{1}$ which is the rate of \cNitrite{} reduction by reduced (active) AniA. This also depends on the concentration of \cNitrite{} and reduced AniA ($A_a$). The reduction of \cNO{} requires $l_1$ which is the rate of reduction of \cNO{} by reduced (active) NorB. This depends on the concentration of \cNO{} and reduced NorB ($B_a$). Inhibition of \cbbthree{} by \cNO{} is modelled by the \engordnumber{3} component of the equation. $k_5$ is the rate of inhibition of \cbbthree{} by \cNO{}. $k_6$ is the rate of recovery of inhibited \cbbthree{}. $\gamma$ is the rate of spontaneous loss of \cNO{} from the electrode chamber.

The reduction of nitrite is modelled by this equation

\begin{equation}
\frac{d[NO_2^-]}{dt} = - m_{1}[NO_2^-][A_a]
\label{eq:nitrite}
\end{equation}
\begin{equation*}
\boxed{
NO_2^- + A_a \xrightarrow{m_1} NO + A_i
}
\end{equation*}


where $m_{1}$ is the rate of reduction of \cNitrite{} by reduced (active) AniA ($A_a$).

\subsubsection{Electron Transporters}

In addition to the rate of change of concentration of the respiratory substrates, the model also contains information about the upstream state of components of the transfer chain, starting from the quinone pool. The ultimate upstream source of electrons into the respiratory chain is from NADH, but I have chosen to subsume all process prior to the quinone pool into a simple single rate. This simplification is made to avoid further complications associated with varying metabolism and to avoid distraction from the stated primary aim of understanding the switching behaviour of the downstream chain. I have chosen the quinone pool as the starting point because it is known that NorB draws electrons directly from this point and therefore this represents the first branch in the chain. I wish understand how competition for electrons at branches effects function and therefore the quinone pool is included in our model.
%In addition to the rate of change of concentration of the respiratory substrates, the model also contains information about the state of the quinone pool, which is the upstream source of electrons into the respiratory chain. This is important because this affects the rate of reduction of the various enzymes which perform the substrate reductions. The equation for modelling the change in reduction state (activity) of the quinone pool is

\begin{equation}
\frac{d[Q_a]}{dt} = g([Q] - [Q_a]) - l_3[Q_a]([B] - [B_a]) - f[Q_a]([X]-[E])
\label{eq:quinones}
\end{equation}
\begin{equation*}
\boxed{
\begin{gathered}
Q_i \xrightarrow{g} Q_a\\
B_i + Q_a \xrightarrow{l_3} B_a + Q_i\\
E_i + Q_a \xrightarrow{f} E + Q_i
\end{gathered}
}
\end{equation*}

$Q_a$ is the reduced quinone, and $Q$ the total concentration of quinones in the system. $g$ represents the constant rate of availability of electrons into the quinone pool from NADH. The reduction of NorB by active quinones is parametrised by the rate constant $l_3$. NorB and reduced NorB are given by $B$ and $B_a$ respectively. As the quinones also reduce the cytochromes, this also needs to be modelled. $f$ is the rate constant parametrising the reduction of cytochromes by the active quinones. Total cytochromes and total reduced cytochromes are given by $X$ and $E$ respectively.

I am using a simplified version of cytochromes and therefore $X$ actually represents a pool of different cytochromes, \textit{c$_{\textrm{x}}$}, \textit{c$_{\textrm{4}}$}, \textit{c$_{\textrm{5}}$} and the \textit{bc$_{\textrm{1}}$} complex. These are amalgamated into one here to simplify the equations and focus on the simple branching of the chain and competition for electrons. This is a modelling choice and it is further discussed in Chapter \ref{chap:completedmodel}.

%$Q_a$ is the reduced quinone, and $Q$ the total concentration of quinones in the system. $g$ represents the rate of flow of electrons into the quinone pool from NADH. The rate of reduction of NorB by active quinones is given by $l_3$. NorB and reduced NorB are given by $B$ and $B_a$ respectively. As the quinones also reduce the cytochromes, this also needs to be modelled. $f$ denotes the rate of reduction of cytochromes by the active quinones. Cytochromes and reduced cytochromes are given by $X$ and $E$ respectively.

%Given that the concentration of active cytochromes changes, due to reduction by the quinone pool and oxidation by the downstream enzymes, and this concentration is a parameter in (\ref{eq:quinones}), it also needs to be included in the model, and this is given by the following equation
The concentration of active cytochrome pool changes due to both reduction by the upstream quinone pool and oxidation by both of the remaining downstream terminal enzymes. 
\begin{equation}
\frac{d[E]}{dt} = -k_3([C] - [C_a] - [C_X])[E]  - m_3([A] - [A_a])[E] + f[Q_a]([X]-[E])
\label{eq:cytochromes}
\end{equation}
\begin{equation*}
\boxed{
\begin{gathered}
C_i + E \xrightarrow{k_3} C_a + E_i\\
A_i + E \xrightarrow{m_3} A_a + E_i\\
E_i + Q_a \xrightarrow{f} E + Q_i
\end{gathered}
}
\end{equation*}

where $k_3$ is the rate constant describing the reduction of the available oxidised cytochrome c oxygen reductase (\cbbthree{}) by the quinone pool (via \textit{c$_{\textrm{x}}$} \& \textit{c$_{\textrm{4}}$}). $C$, $C_a$ and $C_X$ represent the overall concentration of \textit{cbb$_{\textrm{3}}$}, reduced (active) \cbbthree{} and NO inhibited \cbbthree{} respectively. $m_3$ is the rate constant describing the reduction of AniA by the cytochrome pool (via \textit{c$_{\textrm{5}}$}). The concentration of active cytochromes thus increases by their reduction by the quinone pool, but this in turn can reduce the flux from the pool because less oxidised cytochrome is available to accept electrons. As stated previously, the relative time scales are unknown so all processes appear explicitly.
%where $k_3$ is the rate of reduction of the cytochrome c oxygen reductase (\cbbthree{}) by the quinone pool (via \textit{c$_{\textrm{x}}$} \& \textit{c$_{\textrm{4}}$}). $C$, $C_a$ and $C_X$ represent the overall concentration of \cbbthree{}, reduced (active) \cbbthree{} and NO inhibited \cbbthree{} respectively. $m_3$ is the rate of reduction of AniA by the cytochrome pool (via \textit{c$_{\textrm{5}}$}). The concentration of active cytochromes increases by their reduction by the quinone pool.

\subsubsection{Terminal Reductases}
Finally the changes in concentration of reduced terminal oxidases, \cbbthree{}, AniA and NorB are described by the following equations. All the terms present in this section have been introduced previously. I could of course equally write these equations for the oxidised form but this can easily be recovered because I am assuming that the total concentration of the oxidases remains constant.
%To model the changes in concentration of the individual enzymes, \cbbthree{}, AniA and NorB, the following equations are used:

\begin{equation}
\frac{d[C_a]}{dt} = k_3([C] - [C_a] - [C_X])[E] - k_{1}[C_a][O_2] - k_5[C_a][NO]
\label{eq:active_cbb3}
\end{equation}
\begin{equation*}
\boxed{
\begin{gathered}
C_i + E \xrightarrow{k_3} C_a + E_i\\
O_2 + C_a\xrightarrow{k_1} H_{2}O + C_i\\
NO + C_a \xrightarrow{k_5} NO\mhyphen C_X 
\end{gathered}
} 
\end{equation*}


describes the changes in the concentration of reduced (active) \cbbthree{} and

\begin{equation}
\frac{d[C_X]}{dt} = k_5[C_a][NO] - k_6 [C_X]
\label{eq:NO inhibited_cbb3}
\end{equation}

\begin{equation*}
 \boxed{
\begin{gathered}
 NO + C_a \xrightarrow{k_5} NO\mhyphen C_X \xrightarrow{k_6} NO + C_a
\end{gathered}
}
\end{equation*}


described the concentration of \cbbthree{} that has been reversibly inhibited by NO binding.

\begin{equation}
\frac{d[A_a]}{dt} = m_3([A] - [A_a])[E]- m_{1}[NO_2^-][A_a]
\label{eq:active_ania}
\end{equation}

\begin{equation*}
 \boxed{
\begin{gathered}
A_i + E \xrightarrow{m_3} A_a + E_i \\
NO_2^- + A_a \xrightarrow{m_1} NO + A_i 
\end{gathered}
}
\end{equation*}


describes changes in the reduced (active) AniA concentrations and

\begin{equation}
\frac{d[B_a]}{dt} = l_3[Q_a]([B] - [B_a]) - l_1[NO][B_a]\\
\label{eq:active_norb}
\end{equation}
\begin{equation*}
\boxed{
\begin{gathered}
B_i + Q_a \xrightarrow{l_3} Q_i + B_a \\
NO + B_a \xrightarrow{l_1} N_{2}O + B_i
\end{gathered}
}
\end{equation*}


captures the changes in the reduced (active) NorB concentrations.

\subsection{Assumptions and their Justifications}
I have made a number of assumptions regarding the kinetics and reactions taking place in the model.
\begin{enumerate}
 \item {\bf I have assumed that NO inhibits the reduced \cbbthree{} and not the oxidised form, since I wouldn't expect Nitric oxide to bind to an inactive enzyme.} This is corroborated by \citet{Giuffre2000}, who show significant levels of inhibition of reduced cytochrome. They do also however observe low levels of inhibition of the oxidised enzyme also. Their experiments used cytochrome c oxidase (aa3) rather than \cbbthree{}, but I believe this assumption still stands as the enzymes are of the same family.
 \item {\bf Bacterial population structure and concentration variation.} The primary substrates of interest are gases which are thought to freely diffuse in and out of the cells.
 \item {\bf No backwards reactions.}
 \item {\bf No Michaelis-Menton kinetics.} Cannot assume that time-scales are separated as the rates of intermediate reaction steps are not known.
 \item {\bf All cytochromes can be modelled as one.}
 \item {\bf Laz and $c_5$ effects on AniA and \cbbthree{} respectively can be ignored.} They are not the prime electron donors to their terminal reductases and contribute very little overall to the reduction\cite{Deeudom2007}.
\end{enumerate}

%Also, values are in the right ball park even though they are using aa3 rather than \cbbthree. $10^8 M ^{-1} s ^{-1}$ against around $50 \mu M ^{-1} s ^{-1}$.

\section{Parameters and their Prior Distributions}

None of the rate constants or concentrations which were required for this model have previously been determined for \Nsm, so values from other similar organisms had to be used instead. In some cases there appears to be no data in the literature regarding values of particular components. Table \ref{tab:ps} lists the values that have been obtained from the literature.

\begin{table}[tbp]
\begin{center}
\begin{tabular}{ccc}
\toprule
\textbf{Symbol} & \textbf{Description} & \textbf{Value}\\
\midrule
$k_1$ & Rate of O$_{\textrm{2}}$ reduction by reduced \cbbthree{} & $415~\mu M^{-1} s^{-1}$ \\
$k_3$ & Rate of \cbbthree{} reduction by cytochrome pool & $3~\mu M^{-1} s^{-1}$\\ 
$l_1$ & Rate of NO reduction by reduced NorB\\
$l_3$ & Rate of NorB reduction by quinone pool\\
$m_1$ & Rate of NO$_{\textrm{2}}^{\textrm{-}}$ reduction by reduced AniA\\
$m_3$ & Rate of AniA reduction by cytochrome pool & $4.8\pm0.2~\mu M^{-1}s^{-1}$\\
$k_5$ & Rate of \cbbthree{} inhibition by NO & $10^8~M ^{-1} s ^{-1}$\\
$k_6$ & Rate of recovery of NO inhibited \cbbthree{}\\
$\beta$ & Rate of passive diffusion in of O$_{\textrm{2}}$\\
$K_O$ & Saturation O$_{\textrm{2}}$ level & $126~\mu M$\\
$g$ & Rate of electrons in from NADH\\
$f$ & Rate of reduction of cytochromes by quinones\\
$\gamma$ & Spontaneous loss of NO\\
$Q$ & Concentration of quinones & $0.3~\mu M$\\
$X$ & Concentration of cytochromes & 4000~nM - \cbbthree{}\\   
$A$ & Concentration of AniA\\
$B$ & Concentration of NorB\\
$C$ & Concentration of \cbbthree{} & 30~nM \\
\bottomrule
\end{tabular}
\caption{Model parameters
\label{tab:ps}}
\end{center}
\end{table}

\subsection*{Variables}
\subsubsection*{$\mathbf{O_2}$ {\bf- Oxygen concentration}}
This variable is always obtained directly from the experimental dataset as it indicates the starting point for oxygen in the model. It is always set to the first oxygen data point in the dataset and has no prior distribution. It is usually a fixed value, except in cases where the dataset indicates measurement artefacts.

\subsubsection*{$\mathbf{NO}$ {\bf- Nitric oxide concentration}}
As for Oxygen concentration, this variable is simply obtained from the dataset and the same conditions apply.

\subsubsection*{$\mathbf{NO_2^-}$ {\bf- Nitrite concentration}}
Nitrite concentration is also handled in the same way as the oxygen and nitric oxide concentrations.

\subsubsection*{$\mathbf{E}$ {\bf- Reduced cytochrome concentration}}
Unknown at start of simulation. Assume close to zero, and certainly less than total. The best prior to use would be the value after a couple of simulation seconds, however this would no longer be a prior.

\subsubsection*{$\mathbf{A_a}$ {\bf- Reduced AniA}}
Unknown at start of simulation. Assume close to zero, and certainly less than total. The best prior to use would be the value after a couple of simulation seconds, however this would no longer be a prior.

\subsubsection*{$\mathbf{B_a}$ {\bf- Reduced NorB}}
Unknown at start of simulation. Assume close to zero, and certainly less than total. The best prior to use would be the value after a couple of simulation seconds, however this would no longer be a prior.

\subsubsection*{$\mathbf{C_a}$ {\bf- Reduced \cbbthree{}}}
Unknown at start of simulation. Assume close to zero, and certainly less than total. The best prior to use would be the value after a couple of simulation seconds, however this would no longer be a prior.

\subsubsection*{$\mathbf{C_X}$ {\bf- Reversibly NO inhibited \cbbthree{}}}
Unknown at start of simulation. Assume close to zero, and certainly less than total. The best prior to use would be the value after a couple of simulation seconds, however this would no longer be a prior.

\subsubsection*{$\mathbf{Q_a}$ {\bf- Reduced Quinones}}
Unknown at start of simulation. Assume close to zero, and certainly less than total. The best prior to use would be the value after a couple of simulation seconds, however this would no longer be a prior.

\subsection*{Parameters}
\subsubsection*{$\mathbf{k_1}$ {\bf- Rate of O$_{\textrm{2}}$ reduction by reduced \cbbthree{}}}
%\citet{Preisig1996} show that \textit{B. japonicum} \cbbthree{} (\textit{fixNOQP}) has a $K_m$ of $55.7 \pm 24.2$ nM $\mathrm{O}_2$, and $V_{max}$ $37.4 \pm 9.2$ $\mathrm{nmol O}_2 \mathrm{min}^{-1} \mathrm{mg}^{-1}$.\\
%Given $v = \frac{V_{max}[S]}{K_m+[S]}$
%Then at high $O_2$ rate is:
%$v = \frac{37.4\times 100,000}{55.7+100,000} = 37.4~nmol^{-1}min{-1}mg{-1} = 0.000622986~\mu mol^{-1}s{-1}mg{-1}$\\

A value for $k_1$, the rate of O$_{\textrm{2}}$ reduction by reduced \textit{cbb$_{\textrm{3}}$} was calculated by using the $\mathrm{K}_{cat}$ value from \textit{Pseudomonas stutzeri}, and the $\textrm{k}_m$ value from \textit{Neisseria lactamica}, which \citet{Forte2001} and \citet{Hunter2007} determine are $166s^{-1}$ and $0.4\mu M$ respectively. $k_1$ can be calculated as $\frac{166s^{-1}}{0.4\mu M} = 415\mu M^{-1} s^{-1}$.

\subsubsection*{$\mathbf{k_3}$ {\bf- Rate of \cbbthree{} reduction by cytochrome pool}}
$k_3$, the rate of reduction of \textit{cbb$_{\textrm{3}}$} by the cytochromes was calculated from values obtained from the maximum reduction rate of \textit{cbb$_{\textrm{3}}$} by cytochrome $c_4$ in \textit{Vibrio cholerae} by \citet{Chang2010}. A rate of 300 electrons transported per second was observed with a cytochrome $c_4$ concentration of $100\mu M$. This concentration was not saturating, but there appears to be a linear relationship between rate and concentration. We assume that 1 electron equals 1 reduction of \textit{cbb$_{\textrm{3}}$}, thus the rate of reduction of \textit{cbb$_{\textrm{3}}$} by cytochromes is $\frac{300s^{-1}}{100\mu M} = 3\mu M^{-1} s^{-1}$.

\subsubsection*{$\mathbf{l_1}$ {\bf- Rate of NO reduction by reduced NorB}}
240 and 256 nanomoles NO reduced per minute per OD600 unit as determined by \citet{Barth2009}.

This needs fixing. 50\% dry weight, rather than 15\% wet weight.\\
Observed rates of NO reduction by \citet{Rock2007} give $54 \pm 6~\mathrm{nmol min}^{-1} \mathrm{mg}^{-1}$. This is in whole cells however.
$\approx 10~\mathrm{nmol s}^{-1}$ in an $OD_{600} = 1$ culture. Converting to molar gives $2~\mu\mathrm{M s}^{-1}$. 

\subsubsection*{$\mathbf{l_3}$ {\bf- Rate of NorB reduction by quinone pool}}
Benchmark estimate is somewhere around about $1~\mu M^{-1}s^{-1}$.

\subsubsection*{$\mathbf{m_1}$ {\bf- Rate of NO$_{\textrm{2}}^{\textrm{-}}$ reduction by reduced AniA}}


\subsubsection*{$\mathbf{m_3}$ {\bf- Rate of AniA reduction by cytochrome pool}}
The value for $m_3$, the rate of reduction of AniA by cytochromes, is the observed electron transfer rate between the equivalent cytochrome and nitrite reductase from \textit{Achromobacter xylosoxidans}. A value of $4.8\pm0.2 \mu M^{-1}s^{-1}$ was observed during stopped-flow experiments by \citet{Nojiri2009}.

\subsubsection*{$\mathbf{k_5}$ {\bf- Rate of \cbbthree{} inhibition by NO}}
\citet{Giuffre2000} and \citet{Blackmore1991} showed with cytochrome \textit{c} oxidase that NO could bind reversibly and inhibit the activity of the enzyme. The rate they calculated was $10^8~M ^{-1} s ^{-1}$. I assume that even though the enzyme is different, its NO binding characteristics would be similar to that of \cbbthree{} as it is of the same family.

\subsubsection*{$\mathbf{k_6}$ {\bf- Rate of recovery of NO inhibited \cbbthree{}}}
\citet{Giuffre2000} calculated a half-life of t\textonehalf $\approx 80~\mathrm{min}$.\\
$K_d$ from \citet{Rock2007} was calculated to be about 500 nM, which tallies with values from $k_5$ and $k_6$.

\subsubsection*{$\beta$ {\bf- Rate of passive diffusion in of O$_{\textrm{2}}$}}
This value is highly dependent on the culture, and is in some way tied to the density of the culture, however the relationship is not known. During early experiments I noticed that oxygen diffusion was slower in high density cultures compared to those of low density, experiments to examine the relationship proved fruitless in determining any relationship. In addition this parameter is a product of the experimental set-up rather than the model itself.

\subsubsection*{$\mathbf{K_O}$ {\bf- Saturation O$_{\textrm{2}}$ level}}
This value is dependent on the particular culture being modelled, however it's prior value is usually set to $126~\mu M$ as this figure was observed during experiments to determine oxygen diffusion rates into the culture.

\subsubsection*{$\mathbf{g}$ {\bf- Rate of electrons in from NADH (or rate of reduction of quinones)}}


\subsubsection*{$\mathbf{f}$ {\bf- Rate of reduction of cytochromes by quinones}}
\citet{Snyder2000} showed by reducing yeast cytochrome $\mathrm{bc}_1$ by using $25~\mu\mathrm{M}$ menaquinol the rate constants were $7.9~\mathrm{s}^{-1}$ for cytochrome b, and $1.55-6.9\times10^5~\mathrm{M}^{-1}\mathrm{s}^{-1}$ for cytochrome $\mathrm{c}_1$ (second order). $0.155 - 0.69~\mu\mathrm{M}^{-1}\mathrm{s}^{-1}$.

\subsubsection*{$\gamma$ {\bf- Spontaneous loss of NO}}


\subsubsection*{$\mathbf{Q}$ {\bf- Concentration of quinones}}
$Q$, the concentration of quinones was calculated based on data from \citet{Hedrick1986}. The protein content of the cells was assumed to be similar to that of \textit{E. coli} at 15\% of wet weight, where each cell weighed 2pg, and that there were $1\mu \textrm{mol}$ of respiratory quinones per g of bacterial protein. A culture of \textit{Neisseria meningitidis} with $OD_{600} = 1$ has $1 \times 10^9 \textrm{cells/ml}$, therefore there are 1.5nmol of quinones in 5ml culture ($5\times 10^9 \textrm{cells} \times 2\times 10^{-12} \textrm{g} \times 15\% \times 1\mu\textrm{mol/g}$), converted to molarity is $0.3\mu M$.

\subsubsection*{$\mathbf{X}$ {\bf- Concentration of cytochromes}}
\citet{Deeudom2007} suggests total cytochrome concentration (inc. \cbbthree{}) to be about 4000 nM.

\subsubsection*{$\mathbf{A}$ {\bf- Concentration of AniA}}
No idea, probably need to guess based on cell volume (0.6-1.0 $\mu$m diameter, no useful ref), 10\% of cell volume being membrane, and number of proteins in membrane.

\subsubsection*{$\mathbf{B}$ {\bf- Concentration of NorB}}
No idea, probably need to guess based on cell volume (0.6-1.0 $\mu$m diameter, no useful ref), 10\% of cell volume being membrane, and number of proteins in membrane.

\subsubsection*{$\mathbf{C}$ {\bf- Concentration of \cbbthree{}}}
No idea, probably need to guess based on cell volume (0.6-1.0 $\mu$m diameter, no useful ref), 10\% of cell volume being membrane, and number of proteins in membrane.\\
\cbbthree{} is probably 0.1-1\% of cell protein. 10\% of cell is membrane.
$15~\mu g$ in 5 ml based on numbers from Q above. \cbbthree{} is approximately 100 kDa in molecular weight. Converting to molarity gives a concentration of approximately 30 nM.

\section{Implementation of the model}
%By keeping the quantities involved in their original state and not making any assumption about time-scale separation I am able to make predictions regarding the transient oxidation states of the various components. These are potentially experimentally accessible and appear to be crucial for the dynamic response of the chain in different environments.

The model contains no implied information about cell density. This means the values for various component concentrations will differ between experiments. 
Initially the optical density of cultures was used to determine the cell density however experiments proved that this was not a completely reliable proxy for cell density as this also includes dead cells. Using optical density as a cell density proxy should give linear relationships between cell densities and reaction rates, however this proved not to be the case, with rates of oxygen reduction differing between cultures with the same optical density (data not shown). Therefore where possible, any normalisation that was carried out used the initial oxygen reduction rate as a relative indicator of living cells.

%The model contains no implied information about cell density. This means the values for various component concentrations will differ between experiments. Initially the optical density of cultures was used to determine the cell density however experiments proved that this was not a completely reliable proxy for cell density as this also includes dead cells. Using optical density as a cell density proxy should haven given linear relations between cell densities and reaction rates, however this proved not to be the case, with rates of oxygen reduction different between cultures with the same optical density. Therefore where possible, any normalisation that was carried out used the initial oxygen reduction rate as a relative indicator.

\section{Solving Ordinary Differential Equations}
The model equations (given previously) are solved in parallel using the common $\mathrm{6}^\mathrm{th}$ order Runge-Kutta-Fehlberg algorithm for integrating ordinary differential equations\cite{Butcher2003}. Adaptive step-sizes were implemented using the Cash-Karp method\cite{Cash1990}. The adaptive step size system was required as it prevented the introduction of systemic numerical instabilities.

The parameter estimation system and ODE solver were a bespoke implementation written in Java. The Runge-Kutta algorithm was modified from that found in Numerical Recipes in C\cite{Press1992}. I decided to write a custom implementation rather than using off the shelf systems for solving ODEs and parameter estimation as I wanted the greatest flexibility in how I integrated the two techniques, and it allowed me to quickly and easily tailor the code to my needs. Initially I tried using COPASI \cite{Hoops2006}, however at that time it had limitations that I could not overcome, such as an inability to allow bulk addition of components at arbitrary time-points.

The implementation of the model has no constraints on respiratory substrate concentration, thus allows the altering of these concentrations whilst solving the equations, however changes to substrate concentration have to be made programmatically to inform the model of the change (\texttt{if (t == 50) then NO\_conc += 20;}). This ability means that the switch between aerobic and anaerobic respiration can be examined synthetically, and the model is also capable of simulating how the respiratory system responds to the sudden addition of substrates such as Nitric Oxide. More complicated methods are possible, but given the high diffusion of the substrates concerned as well as the deliberate injection of the relevant substrate this method was a simpler and reasonable mimic for my empirical method. This ability was an absolute requirement, as in order to fully parametrise the model it was necessary to isolate sections of the model, which required adding aliquots of respiratory substrate during respiration.

%This ability means that the switch between aerobic and anaerobic respiration can be examined synthetically, and the model is also capable of simulating how the respiratory system responds to addition of substrates such as Nitric Oxide. This ability was an absolute requirement, as in order to fully parametrise the model it was necessary to isolate sections of the model, which required adding aliquots of respiratory substrate during respiration.

\section{Parameter Estimation}
Estimating the parameter values for the components in the mathematical model involved comparing the biological results with those produced by solving the ODEs and adjusting the parameter values to minimise the difference between the two results. The different methods for parameter estimation that I investigated are detailed in Chapter \ref{chap:paramest} [\nameref{chap:paramest}].