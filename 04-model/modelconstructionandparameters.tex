\chapter{Model - Construction and Parameters}
\section{Construction}
\subsection{Converting Biological Reactions into Differential Equations}
Where the reaction is describing a chemical process, the rate constant is given above the arrow, and the relevant enzyme shown in parentheses. Where the reaction is showing the addition of electrons (reduction), this is denoted by $e^-$ below the arrow, the rate constant above, and the source of electrons in parentheses.

The equation that gives the change in oxygen concentration is

\begin{eqnarray}
&\frac{d[O_2]}{dt} = \beta(1-[O_2]/K_O) - k_{1}[C_a][O_2] \nonumber \\\nonumber \\
&\xrightarrow{\beta} \mathbf{O_2} \xrightarrow{k_1~(C_a)} \textnormal{H$_\textnormal{2}$O}
\label{eq:oxygen}
\end{eqnarray}

where $\beta$ is the rate of passive diffusion of \cOxygen \space into the electrode chamber. This is inversely proportional to oxygen concentration in the chamber, and limited to the oxygen saturation concentration, $K_O$. This component of the equation is required to account for a peculiarity of the experimental set-up, whereby the rate of diffusion of oxygen into the system depends on the density of the bacterial culture, and is not insignificant. $k_{1}$ is the rate of reduction of oxygen by the oxygen reductase \textit{cbb$_{\textrm{3}}$}. This rate depends on the concentration of reduced (i.e. active) \textit{cbb$_{\textrm{3}}$}, $C_a$ and the concentration of \cOxygen.

The equation for describing \cNO \space concentration changes is more complex as \cNO \space has a number of additional interactions in comparison to \cOxygen. \cNO \space also interacts with \textit{cbb$_{\textrm{3}}$}, in addition to being reduced from \cNitrite, reduced to \cNtwoO \space and spontaneously lost from the electrode chamber. Currently this is the equation being used to model \cNO \space concentration.

\begin{eqnarray}
&\frac{d[NO]}{dt} = m_{1}[NO_2^-][A_a] - l_1[NO][B_a] - k_5[C_a][NO] + k_6 [C_X] - \gamma[NO] \nonumber \\\nonumber \\
&\textnormal{NO$_\textnormal{2}^\textnormal{-}$} \xrightarrow{m_1~(A_a)} \mathbf{NO} \xrightarrow{l_1~(B_a)} \textnormal{N$_\textnormal{2}$O} \nonumber \\
&\mathbf{NO} + \textnormal{C$_\textnormal{a}$} \xrightarrow{k_5} \textnormal{NO}\textendash \textnormal{C$_\textnormal{X}$} \xrightarrow{k_6} \mathbf{NO} + \textnormal{C$_\textnormal{a}$} \nonumber \\
&\xrightarrow{\gamma} \mathbf{NO}
\label{eq:no}
\end{eqnarray}

The synthesis of \cNO \space is modelled by $m_{1}$ which is the rate of \cNitrite \space reduction by reduced (active) AniA. This also depends on the concentration of \cNitrite \space and reduced AniA ($A_a$). The reduction of \cNO \space requires $l_1$ which is the rate of reduction of \cNO \space by reduced (active) NorB. This depends on the concentration of \cNO \space and reduced NorB ($B_a$). Inhibition of \textit{cbb$_{\textrm{3}}$} by \cNO \space is modelled by the \engordnumber{3} component of the equation. $k_5$ is the rate of inhibition of \textit{cbb$_{\textrm{3}}$} by \cNO. $k_6$ is the rate of recovery of inhibited \textit{cbb$_{\textrm{3}}$}. $\gamma$ is the rate of spontaneous loss of \cNO \space from the electrode chamber.

The reduction of nitrite is modelled by this equation

\begin{eqnarray}
&\frac{d[NO_2^-]}{dt} = - m_{1}[NO_2^-][A_a] \nonumber \\\nonumber \\
&\mathbf{NO_2^-} \xrightarrow{m_1~(A_a)} \textnormal{NO}
\label{eq:nitrite}
\end{eqnarray}

where $m_{1}$ is the rate of reduction of \cNitrite by reduced (active) AniA ($A_a$).

In addition to the rate of change of concentration of the respiratory substrates, the model also contains information about the state of the quinone pool, which is the upstream source of electrons into the respiratory chain. This is important because this affects the rate of reduction of the various enzymes which perform the substrate reductions. The equation for modelling the change in reduction state (activity) of the quinone pool is

\begin{eqnarray}
&\frac{d[Q_a]}{dt} = g([Q] - [Q_a]) - l_3[Q_a]([B] - [B_a]) - f[Q_a]([X]-[E]) \nonumber \\\nonumber \\
&\xrightarrow[e^-]{g} \mathbf{Q_a} \nonumber \\
&\textnormal{B$_\textnormal{i}$} \xrightarrow[e^-]{l_3~(\mathbf{Q_a})} \textnormal{B$_\textnormal{a}$} \nonumber \\
&\textnormal{X-E} \xrightarrow[e^-]{f~(\mathbf{Q_a})} \textnormal{E}
\label{eq:quinones}
\end{eqnarray}

$Q_a$ is the reduced quinone, and $Q$ the total concentration of quinones in the system. $g$ represents the rate of flow of electrons into the quinone pool from NADH. The rate of reduction of NorB by active quinones is given by $l_3$. NorB and reduced NorB are given by $B$ and $B_a$ respectively. As the quinones also reduce the cytochromes, this also needs to be modelled. $f$ denotes the rate of reduction of cytochromes by the active quinones. Cytochromes and reduced cytochromes are given by $X$ and $E$ respectively.

Given that the concentration of active cytochromes changes, due to reduction by the quinone pool and oxidation by the downstream enzymes, and this concentration is a parameter in (\ref{eq:quinones}), it also needs to be included in the model, and this is given by the following equation

\begin{eqnarray}
&\frac{d[E]}{dt} = -k_3([C] - [C_a] - [C_X])[E]  - m_3([A] - [A_a])[E] + f[Q_a]([X]-[E]) \nonumber \\\nonumber \\
&\textnormal{C$_\textnormal{i}$} \xrightarrow[e^-]{k_3~(\mathbf{E})} \textnormal{C$_\textnormal{a}$} \nonumber \\
&\textnormal{A$_\textnormal{i}$} \xrightarrow[e^-]{m_3~(\mathbf{E})} \textnormal{A$_\textnormal{a}$} \nonumber \\
&\textnormal{X-E} \xrightarrow[e^-]{f~(Q_a)} \mathbf{E}
\label{eq:cytochromes}
\end{eqnarray}

where $k_3$ is the rate of reduction of the cytochrome c oxygen reductase (\textit{cbb$_{\textrm{3}}$}) by the quinone pool (via \textit{c$_{\textrm{x}}$} \& \textit{c$_{\textrm{4}}$}). $C$, $C_a$ and $C_X$ represent the overall concentration of \textit{cbb$_{\textrm{3}}$}, reduced (active) \textit{cbb$_{\textrm{3}}$} and denatured \textit{cbb$_{\textrm{3}}$} respectively. $m_3$ is the rate of reduction of AniA by the cytochrome pool (via \textit{c$_{\textrm{5}}$}). The concentration of active cytochromes increases by their reduction by the quinone pool.

To model the changes in concentration of the individual enzymes, \textit{cbb$_{\textrm{3}}$}, AniA and NorB, the following equations are used:

\begin{eqnarray}
&\frac{d[C_a]}{dt} = k_3([C] - [C_a] - [C_X])[E] - k_{1}[C_a][O_2] - k_5[C_a][NO] \nonumber \\\nonumber \\
&\textnormal{C$_\textnormal{i}$} \xrightarrow[e^-]{k_3~(E)} \mathbf{C_a} \nonumber \\
&\textnormal{O$_\textnormal{2}$} \xrightarrow{k_1~(\mathbf{C_a})} \textnormal{H$_\textnormal{2}$O} \nonumber \\
&\textnormal{NO} + \mathbf{C_a} \xrightarrow{k_5} \textnormal{NO}\textendash \textnormal{C$_\textnormal{X}$}
\label{eq:active_cbb3}
\end{eqnarray}

This equation models the concentration of reduced (active) \textit{cbb$_{\textrm{3}}$}, and the following equation models the concentration of \textit{cbb$_{\textrm{3}}$} that has been denatured by \cNO.

\begin{eqnarray}
&\frac{d[C_X]}{dt} = k_5[C_a][NO] - k_6 [C_X] \nonumber \\\nonumber \\
&\textnormal{NO} + \textnormal{C$_\textnormal{a}$} \xrightarrow{k_5} \textnormal{NO}\textendash \mathbf{C_X} \xrightarrow{k_6} \textnormal{NO} + \textnormal{C$_\textnormal{a}$}
\label{eq:denatured_cbb3}
\end{eqnarray}

Reduced (active) AniA concentrations are modelled by this equation

\begin{eqnarray}
&\frac{d[A_a]}{dt} = m_3([A] - [A_a])[E]- m_{1}[NO_2^-][A_a] \nonumber \\\nonumber \\
&\textnormal{A$_\textnormal{i}$} \xrightarrow[e^-]{m_3~(E)} \mathbf{A_a} \nonumber \\
&\textnormal{NO$_\textnormal{2}^\textnormal{-}$} \xrightarrow{m_1~(\mathbf{A_a})} \textnormal{NO}
\label{eq:active_ania}
\end{eqnarray}

and reduced (active) NorB concentrations are modelled by this equation

\begin{eqnarray}
&\frac{d[B_a]}{dt} = l_3[Q_a]([B] - [B_a]) - l_1[NO][B_a] \nonumber \\\nonumber \\
&\textnormal{B$_\textnormal{i}$} \xrightarrow[e^-]{l_3~(Q_a)} \mathbf{B_a} \nonumber \\
&\textnormal{NO} \xrightarrow{l_1~(\mathbf{B_a})} \textnormal{N$_\textnormal{2}$O}
\label{eq:active_norb}
\end{eqnarray}

\subsection{Assumptions and their Justifications}
Assume that NO inhibits Reduced cbb3. It does according to \citet{Giuffre2000}. Also, values are in the right ball park even though they are using aa3 rather than cbb3. $10^8 M ^{-1} s ^{-1}$ against around $50 \mu M ^{-1} s ^{-1}$.
\section{Parameters}